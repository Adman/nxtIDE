
\documentclass[10pt,a4paper]{article}
\usepackage[utf8]{inputenc}
\usepackage[Bjarne]{fncychap}                                                 
\usepackage[parfill]{parskip}

\title{nxtIDE reference manual}
\author{XLC Team}
\date{}

\begin{document}
\maketitle

\vspace{6pt}
{\bf Float}({\it motor}) 
    
    Vypne motor (so zotrvačnosťou).


    

\begin{quote}
    \begin{description}
        
\item[motor] ({\emph{int}}) motor, ktorý chceme zastaviť.

    \end{description}
\end{quote}

 

\vspace{6pt}
{\bf ClearScreen}({\it }) 

    Vyčistí obrazovku.
    
 

\vspace{6pt}
{\bf OnRev}({\it motor, rýchlosť}) 
    
    Nastaví motor, aby išiel dozadu a zapne ho.
    


    

\begin{quote}
    \begin{description}
        
\item[motor] ({\emph{int}}) motor, ktorý chceme spustiť.

\item[speed] ({\emph{int}}) rýchlosť, ktorou pôjde motor od 0 do 100. Záporná rýchlosť zmení smer chodu motora.

    \end{description}
\end{quote}

 

\vspace{6pt}
{\bf Sensor}({\it sensor}) 
    
    Načíta hodnotu z daného senzoru.
    

    

\begin{quote}
    \begin{description}
        
\item[sensor] ({\emph{int}}) senzor, z ktorého chceme čítať.

    \end{description}
\end{quote}

 

\vspace{6pt}
{\bf RectOut}({\it }) 

 

\vspace{6pt}
{\bf SetSensor}({\it }) 

 

\vspace{6pt}
{\bf SetSensorLowspeed}({\it }) 

 

\vspace{6pt}
{\bf TextOut}({\it x, y, text}) 
    
    Vypíše text na obrazovku.
    



    

\begin{quote}
    \begin{description}
        
\item[x] ({\emph{int}}) X-ová pozícia textu.

\item[y] ({\emph{int}}) Y-ová pozícia textu.

\item[text] ({\emph{str}}) Text, ktorý sa má vypísať.

    \end{description}
\end{quote}

 

\vspace{6pt}
{\bf ClearLine}({\it riadok}) 
    
    Vyčistí jeden riadok na obrazovke.
    

    

\begin{quote}
    \begin{description}
        
\item[line] ({\emph{int}}) riadok, ktorý chceme vyčistiť.

    \end{description}
\end{quote}

 

\vspace{6pt}
{\bf SensorUS}({\it sensor}) 

    Načíta hodnotu z lowspeed sensoru (e.g. Ultrasonic). Input port
    by mal byt nastaveny ako lowspeed pred použitím.
    

    

\begin{quote}
    \begin{description}
        
\item[sensor] ({\emph{int}}) senzor, z ktorého chceme čítať.

    \end{description}
\end{quote}

 

\vspace{6pt}
{\bf RotateMotor}({\it motor, rýchlosť, uhol}) 
    
    Otočí motor v špecifikovanom smere na špecifikovanj rýchlosti a špecifikovanom čísle stupňov.




    

\begin{quote}
    \begin{description}
        
\item[motor] ({\emph{int}}) motor we want to rotate

\item[speed] ({\emph{int}}) speed we want to run the motor at, from 0 to 100. Negative value reverses direction.

\item[angle] ({\emph{int}}) number of degrees we want to rotate the motor. Negative value reverses direction.

    \end{description}
\end{quote}

 

\vspace{6pt}
{\bf Off}({\it motor}) 
    
    Vypne motor (bez zotvrvačnosti).
    

    

\begin{quote}
    \begin{description}
        
\item[motor] ({\emph{int}}) motor, ktorý chceme zastaviť.

    \end{description}
\end{quote}

 

\vspace{6pt}
{\bf PlayTone}({\it frekvencia, dĺžka}) 
    
    Prehrá tón.



    

\begin{quote}
    \begin{description}
        
\item[freq] ({\emph{int}}) Frekvencia tónu v Hz.

\item[duration] ({\emph{int}}) Dĺžka prehrávania tónu v milisekundách.

    \end{description}
\end{quote}

 

\vspace{6pt}
{\bf NumOut}({\it x, y, číslo}) 
    
    Vypíše číslo na obrazovku.
 



    

\begin{quote}
    \begin{description}
        
\item[x] ({\emph{int}}) X-ová pozícia textu.

\item[y] ({\emph{int}})  Y-ová pozícia textu.

\item[num] ({\emph{int}}) Číslo, ktorá chceme vypísať.

    \end{description}
\end{quote}

 

\vspace{6pt}
{\bf LineOut}({\it }) 

 

\vspace{6pt}
{\bf CircleOut}({\it }) 

 

\vspace{6pt}
{\bf MotorTachoCount}({\it motor}) 
    
    Načítavanie otáčok motora.


    

\begin{quote}
    \begin{description}
        
\item[motor] ({\emph{int}}) motor, z ktorého chceme načítavať.

    \end{description}
\end{quote}

 

\vspace{6pt}
{\bf PointOut}({\it x, y}) 
    
    Nakreslí bod na obrazovke na pozíciach (x, y)



    

\begin{quote}
    \begin{description}
        
\item[x] ({\emph{int}}) X-ová pozícia bodu

\item[y] ({\emph{int}}) Y-ová pozícia bodu

    \end{description}
\end{quote}

 

\vspace{6pt}
{\bf SetSensorLight}({\it }) 

 

\vspace{6pt}
{\bf Random}({\it n = 0}) 
    
    Vráti náhodné číslo
    

    

\begin{quote}
    \begin{description}
        
\item[n] ({\emph{int}}) najväčšia hodnota , ktorú má táto funkcia vrátiť. 

    \end{description}
\end{quote}

 

\vspace{6pt}
{\bf OnFwd}({\it motor, rýchlosť}) 
  
    Nastaví motor aby išiel dopredu, a pustí ho
    


    

\begin{quote}
    \begin{description}
        
\item[motor] ({\emph{int}}) motor, ktorý chceme spustiť.

\item[speed] ({\emph{int}}) Rýchlosť, ktorou chceme poslať robota dopredu od 0 do 100. Záporná rýchlosť zmení smer chodu motora.   

    \end{description}
\end{quote}

 

\vspace{6pt}
{\bf ResetTachoCount}({\it motor}) 
    
    Zresetuje tachometer. 


    

\begin{quote}
    \begin{description}
        
\item[motor] ({\emph{int}}) motor, ktorého tachometer chceme zresetovať.

    \end{description}
\end{quote}

 

\vspace{6pt}
{\bf SetSensorType}({\it }) 

 

\vspace{6pt}
{\bf Wait}({\it milisekundy}) 
    
    Počká na daný čas v milisekundách.


    

\begin{quote}
    \begin{description}
        
\item[milisec] ({\emph{int}}) číslo v milisekundách.

    \end{description}
\end{quote}

 


\end{document}
