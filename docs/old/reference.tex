
\documentclass[10pt,a4paper]{article}
\usepackage[utf8]{inputenc}
\usepackage[Bjarne]{fncychap}                                                 
\usepackage[parfill]{parskip}

\title{nxtIDE reference manual}
\author{XLC Team}
\date{}

\begin{document}
\maketitle

\vspace{6pt}
{\bf Float}({\it motor}) 
    
    Kill power for the motor.





\begin{quote}
    \begin{description}
        
\item[motor] ({\emph{int}}) motor we want to stop.

    \end{description}
\end{quote}

 

\vspace{6pt}
{\bf ClearScreen}({\it }) 
    
    Clear the screen.


 

\vspace{6pt}
{\bf OnRev}({\it motor, speed}) 
    
    Set motor to reverse direction and turn it on.
    





\begin{quote}
    \begin{description}
        
\item[motor] ({\emph{int}}) motor we want to run.

\item[speed] ({\emph{int}}) speed we want to run the motor at from 0 to 100. Negative value reverses direction.

    \end{description}
\end{quote}

 

\vspace{6pt}
{\bf Sensor}({\it sensor}) 
    
    Read value from given sensor.
    




\begin{quote}
    \begin{description}
        
\item[sensor] ({\emph{int}}) sensor we want to read from

    \end{description}
\end{quote}

 

\vspace{6pt}
{\bf RectOut}({\it x, y, width, height}) 
    
    Draw a rectangle from [x, y] with specified width and height.








\begin{quote}
    \begin{description}
        
\item[x] ({\emph{int}}) X coordinate of the start point of the rectangle.

\item[y] ({\emph{int}}) Y coordinate of the start point of the rectangle.

\item[width] ({\emph{int}}) The width of the rectangle.

\item[height] ({\emph{int}}) The height of the rectangle.

    \end{description}
\end{quote}

 

\vspace{6pt}
{\bf SetSensor}({\it sensor, type}) 

 

\vspace{6pt}
{\bf SetSensorLowspeed}({\it sensor}) 

 

\vspace{6pt}
{\bf TextOut}({\it x, y, text}) 
    
    Print text on the screen.
    






\begin{quote}
    \begin{description}
        
\item[x] ({\emph{int}}) X coordinate of the text

\item[y] ({\emph{int}}) Y coordinate or the text

\item[text] ({\emph{str}}) The text to print

    \end{description}
\end{quote}

 

\vspace{6pt}
{\bf ClearLine}({\it line}) 
    
    Clear one line on the screen.
    




\begin{quote}
    \begin{description}
        
\item[line] ({\emph{int}}) line we want to clear.

    \end{description}
\end{quote}

 

\vspace{6pt}
{\bf SensorUS}({\it sensor}) 

    Read value from given lowspeed sensor (e.g. Ultrasonic). The input port 
    has to be configured as a Lowspeed before using this function.
    




\begin{quote}
    \begin{description}
        
\item[sensor] ({\emph{int}}) sensor we want to read from

    \end{description}
\end{quote}

 

\vspace{6pt}
{\bf RotateMotor}({\it motor, speed, angle}) 
    
    Rotate motor in specified direction at specified speed for the specified
    number of degrees.




    


\begin{quote}
    \begin{description}
        
\item[motor] ({\emph{int}}) motor we want to rotate

\item[speed] ({\emph{int}}) speed we want to run the motor at, from 0 to 100. Negative value reverses direction.

\item[angle] ({\emph{int}}) number of degrees we want to rotate the motor. Negative value reverses direction.

    \end{description}
\end{quote}

 

\vspace{6pt}
{\bf Off}({\it motor}) 
    
    Turn the motor off (with break).
    




\begin{quote}
    \begin{description}
        
\item[motor] ({\emph{int}}) motor we want to stop.

    \end{description}
\end{quote}

 

\vspace{6pt}
{\bf PlayTone}({\it freq, duration}) 
    
    Play a tone.






\begin{quote}
    \begin{description}
        
\item[freq] ({\emph{int}}) Frequency of the tone in Hz.

\item[duration] ({\emph{int}}) For how long should the brick play this tone.

    \end{description}
\end{quote}

 

\vspace{6pt}
{\bf NumOut}({\it x, y, num}) 
    
    Print number on the screen.
 






\begin{quote}
    \begin{description}
        
\item[x] ({\emph{int}}) X coordinate of the text

\item[y] ({\emph{int}}) Y coordinate or the text

\item[num] ({\emph{int}}) The number to print

    \end{description}
\end{quote}

 

\vspace{6pt}
{\bf LineOut}({\it x0, y0, x1, y1}) 
    
    Draw a line from [x0, y0] to [x1, y1].








\begin{quote}
    \begin{description}
        
\item[x0] ({\emph{int}}) X coordinate of the start point of the line

\item[y0] ({\emph{int}}) Y coordinate of the start point of the line

\item[x1] ({\emph{int}}) X coordinate of the end point of the line

\item[y1] ({\emph{int}}) Y coordinate of the start point of the line

    \end{description}
\end{quote}

 

\vspace{6pt}
{\bf CircleOut}({\it x, y, radius}) 

    Draw a circle with center at [x, y] and specified radius.







\begin{quote}
    \begin{description}
        
\item[x] ({\emph{int}}) X coordinate of the center of the circle.

\item[x] ({\emph{int}}) Y coordinate of the center of the circle.

\item[radius] ({\emph{int}}) The radius of the circle.

    \end{description}
\end{quote}

 

\vspace{6pt}
{\bf MotorTachoCount}({\it motor}) 
    
    Get motor tachometer counter value.





\begin{quote}
    \begin{description}
        
\item[motor] ({\emph{int}}) motor we want to get tachometer count from.

    \end{description}
\end{quote}

 

\vspace{6pt}
{\bf PointOut}({\it x, y}) 
    
    Draw a point on the screen at (x, y)



    


\begin{quote}
    \begin{description}
        
\item[x] ({\emph{int}}) The x coordinate of the point

\item[y] ({\emph{int}}) The y coordinate of the point

    \end{description}
\end{quote}

 

\vspace{6pt}
{\bf SetSensorLight}({\it sensor}) 

 

\vspace{6pt}
{\bf Random}({\it n = 0}) 
    
    Returns a random number





\begin{quote}
    \begin{description}
        
\item[n] ({\emph{int}}) the maximal value this function should return

    \end{description}
\end{quote}

 

\vspace{6pt}
{\bf OnFwd}({\it motor, speed}) 
  
    Set motor to forward direction and turn it on.
    





\begin{quote}
    \begin{description}
        
\item[motor] ({\emph{int}}) motor we want to run.

\item[speed] ({\emph{int}}) speed we want to run the motor at from 0 to 100. Negative value reverses direction.   

    \end{description}
\end{quote}

 

\vspace{6pt}
{\bf ResetTachoCount}({\it motor}) 
    
    Reset tachometer counter. 





\begin{quote}
    \begin{description}
        
\item[motor] ({\emph{int}}) desired motor output.

    \end{description}
\end{quote}

 

\vspace{6pt}
{\bf SetSensorType}({\it sensor, type}) 

 

\vspace{6pt}
{\bf Wait}({\it milisec}) 
    
    Waits for given number of miliseconds.





\begin{quote}
    \begin{description}
        
\item[milisec] ({\emph{int}}) number of miliseconds

    \end{description}
\end{quote}

 


\end{document}
